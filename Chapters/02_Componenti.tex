\chapter{Componenti di un sistema operativo}
Un sistema operativo offre un ambiente on cui eseguire i programmi e fornire servizi che naturalmente variano in base al sistema operativo. Si possono comunque identificare alcune classi di servizi comuni.
\section{Servizi di gestione}
\subsubsection{Gestione dei processi}
Si intende per processo un programma in esecuzione che necessita di risorse e viene eseguito in modo sequenziale (un'istruzione alla volta). Si differenzia tra processi del sistema operativo e quelli utente. Il 
sistema operativo \`e responsabile della creazione, distruzione, sospensione, riesumazione e ella fornitura di meccanismi per la sincronizzazione e la comunicazione tra processi e fornisce meccanismi per la sincronizzazione. 
\subsubsection{Gestione della memoria primaria}
La memoria primaria conserva dati condivisi dalla CPU e dai dispositivi di I/O. Un programma deve essere caricato in memoria prima di poter essere eseguito. Il sistema operativo \`e responsabile della gestione 
dello spazio di memoria (quali parti e da chi sono usate), ovvero della decisione su quale processo caricare in memoria in base allo spazio disponibile e dell'allocazione e rilascio dello spazio di memoria. 
\subsubsection{Gestione della memoria secondaria}
Essendo la memoria primaria volatile e piccola si rende necessaria una memoria secondaria per mantenere grandi quantit\`a di dati in modo permanente. \`E formata tipicamente da un insieme di dischi 
magnetici (che stanno per essere sostituiti dai dischi a stato solido -SSD- pi\`u veloci e performanti). Il sistema operativo \`e responsabile della gestione dello spazio libero su disco, dell'allocazione dello spazio su disco e 
dello scheduling degli accessi su disco. 
\subsubsection{Gestionde dell'I/O}
Il sistema operativo nasconde all'utente le specifiche caratteristiche dei dispositivi di I/O per motivi di efficienza e protezione. Viene impegato un sistema per accumulare gli accessi ai dispositivi (buffering), una generica 
interfaccia verso i device driver, con device driver specifici per alcuni dispositivi. 
\subsubsection{Gestione dei file}
Le informazioni sono memorizzate su supporti fisici diversi controllati da driver con caratteristiche diverse. Si crea pertanto un file, un'astrazione logica per rendere conveniente l'uso di memoria non volatile 
grazie alla raccolta di informazioni correlate. Il sistema operativo \`e responsabile della creazione e cancellazione di file e directory, del supporto di operazioni primitive per la loro gestione (copia, incolla, modifica), 
della corrispondenza tra file e spazio fisico su disco e del salvataggio delle informazioni a scopo di backup. 
\subsubsection{Protezione}
Si intende con protezione un meccanismo per controllare l'accesso alle risorse da parte di utenti e processi. Il sistema operativo deve definire quali sono gli accessi autorizzati e quali no, i controlli da imporre e fornire gli 
strumenti per verificare le politiche di accesso. La sicurezza di un sistema operativo comincia con l'obbligo di identificazione di ciascun utente che permette l'accesso alle risorse.  
\subsubsection{Rete (sistemi distribuiti)}
Si intende per sistema distribuito una collezione di elementi di calcolo che non condividono n\`e la memoria n\`e un clock: le risorse di calcolo vengono connesse tramite una rete. Il sistema operativo \`e 
responsabile della gestione in rete delle varie componenti.
\section{Interprete dei comandi (shell)}
Vi sono due modi fondamentali per gli utenti di comunicare con il sistema operativo: un primo basato su un'interfaccia a riga d comando (o interprete dei comandi) e un secondo bastato su un'interfaccia grafica o GUI. Il primo lascia
inserire direttamente agli utenti le istruzioni che il sistema deve eseguire. L'interprete dei comandi \`e pi\`u comunemente conosciuto come shell. La funzione principale dell'interprete \`e quella di prelevare ed eseguire il successivo
comando impartito dall'utente. A questo livello si utano nuovi comandi per la gestione dei file che possono essere implementati internamente all'interprete o attraverso programmi speciali. Nel secondo caso l'interprete non capisce il
comando in s\`e ma prende il nome per caricare l'opportuno file in memoria per eseguirlo.
\subsubsection{Interfaccia grafica}
L'interfaccia grafica \`e una modalit\`a di comunicazione tra utente e il sistema operativo. \`E pi\`u intuitiva della riga di comando e la GUI \`e l'equivalente del desktop e rimane strettamente legata a mouse, tastiera e schermo.
Puntando le icone col mouse \`e possibile accedere a file, cartelle e applicazioni. 
\subsection{System calls}
Le chiamate di sistema costituiscono l'interfaccia di comunicazione tra il processo e il sistema operativo. Sono tipicamente scritte in linguaggi di alto livello come \emph{C} o \emph{C++}. I programmatori non si devono preoccupare dei
dettagli di implementazione delle \emph{sys.call} in quanto solitamente utilizzano un'API (application program interface) che specifica un'insieme di funzioni a disposizione dei programmatori e dettaglia i parametri necessari 
all'invocazione di queste funzioni e i valori restituiti. Le due API pi\`u comuni sono \emph{win32} e \emph{POSIX API}, rispettivamente per Windows e UNIX. I parametri delle system calls possono essere pasati per valore o riferimento, ma
vanno fisicamente messi da qualche parte: vengono pertanto posizionati nei registri (molto veloci, ma pochi e di dimensione fissa) nello stack del programma o in una tabella di memoria il cui indirizzo \`e passato in un registro o nello
stack.
